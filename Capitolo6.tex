\large{
Di seguito saranno viste le varie operazioni eseguibili dall'utente sull'editor ad eccezione per la riduzione da grafo clusterizzato a grafo flat a cui è stato riservato il capitolo successivo. Per comodità per il lettore esse saranno divise nelle tre sezioni di seguito riportate:
\begin{itemize}
	\item \textbf{creazione} di un oggetto, di un grafo clusterizzato mediante file esterni o templates;
	\item \textbf{modifica} di un oggetto, aggiunta di informazioni o sostituzione di un attributo;
	\item \textbf{navigazione} all'interno del grafo creato; 
\end{itemize}
Prima di passare alla descrizione dettagliata delle varie operazioni è necessario un focus dettagliato sull'operazione di disegno del grafo nella graph-view in quanto questa è l'operazione che il sistema esegue ogni qual volta venga creato o modificato un oggetto sia della struttura dati che della sua rappresentazione.
\section{disegno dei dati}
Quasi tutte le operazioni di creazione o di modifica portano ad un cambiamento importante dei dati e quindi si ha la necessità di un costante cambiamento della loro visualizzazione. Ogni volta che l'utente esegue un cambiamento o l'aggiunta di un oggetto come mostrato schematicamente nella \figurename~\ref{fig:redraw} il sistema passerà ad effettuare una operazione di disegno dei dati per poi dare di nuovo il controllo all'utente.
\begin{figure}[!htb]
	\begin{center}
		\includegraphics[width=1 \linewidth]{figure/redraw}
	\end{center}
	\caption{Schema dell'impiego della funzione di disegno dei dati \label{fig:redraw}}
\end{figure}
Quando questa operazione viene eseguita il sistema elimina completamente la sola visualizzazione degli svg lasciando però inalterato l'svg principale \#cgraph e ridisegnando con i cambiamenti effettuati nuovamente i dati creati o importati. Ricordando poi che si sta operando con un modello Node-link con una rappresentazione spring-embedding una volta ridisegnati i dati si passerà all'aggiunta delle forse fisiche che controlleranno il movimento e la rappresentazione degli oggetti. In particolare questo viene eseguito mediante le funzionalità di D3. Ogni cluster di livello $l$ avrà dunque:
\begin{itemize}
	\item una forza attrattiva direzionata verso il centro dello stesso esclusivamente per i nodi prenenti nel suo attributo \textit{nodes};
	\item una forza repulsiva verso gli altri cluster dello stesso livello $l$; 
	\item una forza attrattiva direzionata verso il centro dello stesso esclusivamente per i cluster di livello inferiore presenti nel suo attributo \textit{cildren};
\end{itemize}
Ogni nodo invece avrà:
\begin{itemize}
	\item una forza attrattiva direzionata verso il centro del cluster di appartenenza;
	\item una forza repulsiva verso gli altri nodi;
\end{itemize}
Queste forse sono mostrate schematicamente nella figura\figurename~\ref{fig:springExample} in cui ogni vettore rappresenta la forza attrattiva nel caso in cui il verso sia entrante o repulsiva nel caso in cui il verso sia uscente.\\
Una volta definite le forze ed utilizzate l'algoritmo di ridisegno termina lasciando all'utente la possibilità di continuare la sessione di lavoro.
\begin{figure}[!htb]
	\begin{center}
		\includegraphics[width=1 \linewidth]{figure/springExample}
	\end{center}
	\caption{Esempio delle forze attrattive e repulsive utilizzate in fase di disegno \label{fig:springExample}}
\end{figure}
Una volta definita l'operazione di disegno degli oggetti da visualizzare è ora possibile procedere al chiarimento delle funzioni utente negli ambiti della creazione e della modifica degli elementi e della navigazione all'interno della visualizzazione.
\section{creazione}

\subsection{import/export}
Iniziando una sessione di lavoro l'utente ha la possibilità come già accennato di cominciare a creare e rappresentare un grafo clusterizzato oppure di importare dei dati per poterne avere solo la loro visualizzazione. In particolare come visto l'import può essere eseguito mediante dati definibili tabellari provenienti da file con estensione \textit{.Json}.\\
%%% CITARE JSON.COM O ROBA SIMILE !!!
Si ricorda, prima di procedere, che il JSON (JavaScript Object Notation) è un formato di scambio dati di facile lettura e scrittura. Inoltre risulta essere di facile analisi per il calcolatore e di facile generazione. Si basa su un sottoinsieme dello standard di programmazione JavaScript ECMA-262 3a edizione - dicembre 1999. JSON è un formato di testo che è completamente indipendente dal linguaggio ma utilizza convenzioni familiari ai programmatori della famiglia di linguaggi C, tra cui C, C ++, C \#, Java, JavaScript, Perl, Python e molti altri. Queste proprietà rendono JSON un linguaggio di scambio dati ideale.\\
JSON è costruito su due strutture:
\begin{itemize}
	\item Una raccolta di coppie nome / valore. In varie lingue, questo viene realizzato come oggetto, record, struct, dizionario, tabella hash, elenco con chiave o array associativo;
	\item Un elenco ordinato di valori. Nella maggior parte delle lingue, questo è realizzato come una matrice, un vettore, un elenco o una sequenza. 
\end{itemize} 
Mediante un bottone di import è dunque possibile caricare un file json personale creato secondo la struttura fissa mostrata nella \figurename~\ref{fig:json}e passare questi dati al sistema per crearne la rappresentazione associata.\\
\begin{figure}[!htb]
	\begin{center}
		\includegraphics[width=1 \linewidth]{figure/json}
	\end{center}
	\caption{Esempio di file json per l'import di un grafo clusterizzato \label{fig:json}}
\end{figure}
In qualunque momento durante una sessione di lavoro è inoltre possibile l'export del grafo su cui si sta lavorando. In particolare l'utente ha due possibilità di salvataggio: della struttura o della visualizzazione. La prima e più importante è riferita all'export dei dati mediante file json che potranno essere poi ricaricati e riutilizzati per una sessione di lavoro futura. La seconda invece è riferita alla possibilità di salvaggio della rappresentazione dei dati mediante file con estenzione \textit{.PNG}. Si ricorda inoltre che il PNG (Portable Network Graphics) è un formato di file per memorizzare immagini. 
Ogni qualvolta che si sceglie di utilizzare l'import di un file esterno il sistema andrà prima ad eliminare tutto ciò che fa parte della sessione di lavoro su cui l'utente sta lavorando. In questo modo il sistema potrà importare i dati richiesti e procedere con la rappresentazione degli stessi.
\subsection{template}
Non avendo a disposizione file json per l'import dei dati o volendo semplicemente cominciare una sessione di lavoro non da una semplice pagina bianca l'utente ha a disposizione dei modelli predefiniti su cui potrà cominciare il lavoro ed andare a modificare a piacimento. I modelli non rappresentano solamente la visualizzazione in se ma si basano su dati tabellari. Esattamente come per l'import di file json anche l'utilizzo di modelli con un algoritmo simile a quello mostrato nella \figurename~\ref{fig:viewAlg}. In altre parole ogni qual volta l'utente chiede al sistema di creare un modello questo risponde eliminando tutta la sessione di lavoro ed inizializzando nuovamente con i dati definiti durante la richiesta di creazione mediante il modello scelto. Di default alla creazione della pagina di lavoro il sistema presenterà, come visto nella \figurename~\ref{fig:interfaccia}, una pagina di lavoro vuota proprio per lasciare all'utente la scelta non solo di tipologia di dati da poter creare ma anche della loro posizione sul piano di lavoro. Questo andrà però a pregiudicare il fatto che il conseguimento di una buona visualizzazione sarà compito dell'utente.
\subsection{elemento del grafo}
A prescindere dalla scelta di importare dati, utilizzare modelli o cominciare da un grafo vuoto, l'utente avrà la possibilità di inserire cluster, nodi e archi. È consigliabile seguire un criterio, quando è possibile, nella creazione degli oggetti. Un buon metodo di creazione è quello che applica una strategia di discesa dell'albero di inclusione top-down partendo dunque dalla creazione dei cluster di livello uno, fino ad arrivare a quelli più in profondità per poi passare alla creazione delle foglie dell'albero, ovvero dei nodi terminando con gli archi. Ovviamente questa non è una regola e non pregiudica la creazione di un grafo ma è un buon principio per la visualizzazione che un utente dovrebbe seguire. Per questo ed altri motivi sono stati realizzati dei messaggi di errore che aiutano l'utente e lo indirizzano verso l'approccio sopra definito. Supponendo ad esempio che l'utente abbia creato due cluster di livello uno ed un nodo all'interno di uno di essi e dia al sistema l'input per poter cominciare la creazione di archi esso risponderà con il messaggio di errore mostrato in \figurename~\ref{fig:erroreArco}.
\begin{figure}[!htb]
	\begin{center}
		\includegraphics[width=0.8 \linewidth]{figure/erroreArco}
	\end{center}
	\caption{Messaggio di errore per il tentativo di creazione di un arco senza due nodi\label{fig:erroreArco}}
\end{figure}
Ogni volta che l'utente creerà un oggetto all'interno del piano di lavoro il sistema risponderà prima aggiornando la struttura dati a cui la visualizzazione fa riferimento ed andrà poi a creare nuovamente la loro rappresentazione. Una volta creati i cluster di un livello l'utente potrà creare i figli degli stessi. Recepita questa richiesta il sistema andrà ad aggiornare le strutture dati inizializzando il nuovo oggetto e collegandolo, con gli attributi visti nella \figurename~\ref{fig:clusterClass}, al cluster genitore. Un cluster inoltre nella graph view ha un raggio $r_c$ dipendente dal numero di oggetti al suo interno. È possibile definire questo raggio come: 
$$r_c=k_c*( f+1 )+2*n$$ 
$$\forall n>=5$$
$$ \forall f>=0$$
con $k_c$ definito come un valore di default,$n$ come il numero di nodi all'interno del cluster e $f$ come il numero di figli del cluster. Per quanto riguarda i nodi invece essi saranno rappresentati con un raggio fisso $k_n$ non avendo al loro interno oggetti.
Infine per quando riguarda la creazione degli archi all'utente basterà selezionare un nodo che sarà definito sorgente e un nodo destinazione per collegarli. Si è scelto di non utilizzare linee dritte nella rappresentazione ma curvate di modo da evitare qualora possibile intersezioni tra essi come si concerne ad un grafo planare anche se è a discrezione dell'utente il poter definire un grafo in cui gli archi possono incrociarsi o meno. Nella figura \figurename~\ref{fig:archi} è mostrato l'impiego di un gran numero di archi che tra loro non vanno ad intersecarsi al contrario di come sarebbe successo utilizzando linee rette.
\begin{figure}[!htb]
	\begin{center}
		\includegraphics[width=0.8 \linewidth]{figure/archi}
	\end{center}
	\caption{Esempio di planarità nella rappresentazione di un grafo clusterizzato\label{fig:archi}}
\end{figure}

\section{modifica}
Il sistema lascia una buona libertà all'utente per quanto concerne la modifica degli oggetti le cui operazioni saranno viste nel dettaglio di seguito. \\
\subsection{Spostamento e cancellazione}
Per quanto riguarda gli oggetti visualizzati, che sono stati creati o importati durante la sessione di lavoro, l'utente ha la possibilità di spostamento sia di un oggetto cluster che di un nodo. Inoltre il sistema spostando ad esempio un nodo, durante l'operazione successiva provvederà a spostare gli archi ad esso collegati in maniera automatica anche se un nodo può essere mosso solamente all'interno del cluster di appartenenza poiché esso non cambierà la struttura dati alla base ma solamente le sue coordinate nella visualizzazione.
Essendo l'utente finale non esente da possibili errori di creazione degli oggetti è stata realizzata una funzione per la cancellazione di oggetti. Al contrario che nelle funzioni di spostamento di un oggetto in questo caso si avranno effetti diversi a seconda dell'oggetto cancellato, che sarà eliminato anche dalla struttura dati alla base della rappresentazione, in sintesi eliminando:
\begin{itemize}
	\item un nodo: verrà eliminata la sua visualizzazione, quella di tutti gli archi che avevano l'id di quel nodo come valore \textit{source} o come valore \textit{target} e l'id del nodo verrà eliminato dalla lista dei nodi del cluster di appartenenza;
	\item un cluster: verrà eliminata la sua visualizzazione,tutti i nodi con l'id uguale a quelli appartenenti al cluster e nel caso in cui il cluster fosse di livello $l>1$ allora sarà eliminato dalla lista dei figli del cluster genitore e ridotto il raggio dello stesso essendo dipendente anche dal numero di figli.
\end{itemize}
Si riporta, in maniera semplificata, nella \figurename~\ref{fig:delete} l'algoritmo di eliminazione di un oggetto.
\begin{figure}[!htb]
	\begin{center}
		\includegraphics[width=0.8 \linewidth]{figure/delete}
	\end{center}
	\caption{algoritmo di cancellazione di un oggetto\label{fig:delete}}
\end{figure}
\subsection{raggio e colore degli oggetti}
Oltre alle operazioni di cancellazione e spostamento l'utente ha la libertà di assegnare valori diversi da quelli di default per colori e raggio agli oggetti del grafo. In particolare, ricordando che il raggio di un cluster è definito come un valore di default $k_c$ moltiplicato per il numero e all'entità degli oggetti mentre quello di un nodo è un semplice valore $k_n$, l'utente potrà andare a modificare la dimensione di questi due valori indicando al sistema la lunghezza desiderata in pixel del raggio. Il sistema applicherà la modifica a tutti gli oggetti della categoria dando un avvertimento all'utente nel caso in cui verrà inserito un valore eccessivamente grande per $k_c$ o $k_n$. Avendo la dimensione dello spazio di lavoro fissa e dipendente dal proprio calcolatore, quella di poter cambiare i valori standard di un oggetto andrà si a modificarne la visualizzazione ma soprattutto a poter avere un maggiore spazio a disposizione nel caso in cui il grafo raggiunge dimensioni considerevoli ma anche di avere un maggior dettaglio nel caso inverso in cui un grafo presenta ad esempio un numero limitato di cluster ed un significativo numero di nodi per ognuno di essi. Ad esempio lo stesso grafo clusterizzato riportato nella \figurename~\ref{fig:graphView} andando a dimezzare la dimensione solamente dei suoi cluster e eseguendo qualche piccolo accorgimento per quanto concerne lo spostamento degli oggetti come è riportato nella figura \figurename~\ref{fig:changeRad} come sia possibile avere a disposizione molto più spazio anche lavorando su un piano di lavoro delle stesse dimensioni di quello utilizzato in precedenza.
\begin{figure}[!htb]
	\begin{center}
		\includegraphics[width=0.8 \linewidth]{figure/changeRad}
	\end{center}
	\caption{operazione di cambiamento del raggio dei cluster da parte dell'utente\label{fig:changeRad}}
\end{figure}
Come già accennato in precedenza ai cluster è assegnato un colore. Questo è di notevole importanza essendo un principio fondamentale per la rappresentazione come espresso anche nel capitolo sulle primitive della visualizzazione. Al colore di ogni cluster di default viene attribuito un valore random una volta che esso è stato aggiunto all'oggetto \textit{clusteredgraph} e deve esser rappresentato. L'utente può cambiare questa tipologia di colorazione in due modi di seguito illustrati.\\
Il primo è quello di cambiare palette di lavoro avendo a disposizione tre colorazioni diverse che si riferiscono a tutte le sfumature facenti parte di quella particolare tonalità. Quando l'utente sceglierà di lavorare su una precisa tonalità per il prossimo numero indefinito di cluster il sistema imposterà alcuni valori esadecimale fissi per poter avere una scala di colori che richiami solo la tonalità richiesta dall'utente. Riprendendo nuovamente il grafo mostrato nella figura \figurename~\ref{fig:graphView} si può notare come essendo completamente casuale possono capitare anche colori non appariscenti o comunque non molto gradevoli all'occhio umano per questo l'utente mediante questa funzione potrà, a puro titolo di esempio, trasformarlo nel grafo mostrato nella \figurename~\ref{fig:changePalette}.
\begin{figure}[!htb]
	\begin{center}
		\includegraphics[width=0.8 \linewidth]{figure/changePalette}
	\end{center}
	\caption{impiego di Palette diverse da quella di default da parte dell'utente\label{fig:changePalette}}
\end{figure}
In questo modo è possibile anche catalogare e categorizzare i cluster in base alla loro colorazione.
Per concludere la sezione sulle possibili operazioni di modifica inerenti al raggio e alla colorazione degli oggetti l'utente ha la possibilità di sostituire il colore assegnato ad un singolo cluster. In questo caso il sistema chiederà all'utente di inserire il valore esadecimale scelto per il colore. Una volta inserito verrà chiesto all'utente di decidere l'oggetto a cui applicare tale modifica. Si ricorda che il colore può essere definito anche con un valore esadecimale come visto anche nel paragrafo inerenti alle primitive per la visualizzazione.
\subsection{aggiungere descrizione agli oggetti}
L'editor è predisposto per lavorare su un numero moderatamente grande di cluster e di oggetti in generali. Per questo può esser utile aggiungere sul piano di lavoro etichette o comunque descrizioni degli oggetti che sono stati rappresentati. A titolo di esempio si può porre il caso in cui ogni cluster debba rappresentare un particolare oggetto o comunque si abbia bisogno un aiuto descrittivo su cosa rappresenti quell'oggetto per l'utente che andrà a realizzarlo o ancora bisogna lasciare dei brevi commenti er essere visualizzati tra una sessione di lavoro e l'altra sullo stesso cluster. Questi problemi sono stati risposti dando la possibilità all'utente di poter richiedere al sistema un spazio in cui scrivere queste descrizioni o aiuti specifici per un particolare oggetto. Una volta scritto il commento il sistema chiederà all'utente di selezionare l'oggetto a cui questa stringa sarà legata. Una volta visualizzata, la posizione all'interno della visualizzazione della stringa dipenderà dalle coordinate dell'oggetto associato di modo che in caso di modifica, spostamento o cancellazione la posizione del testo verrà modificata o sarà eliminato.\\
Un esempio di utilizzo, facendo ancora una volta riferimento alla \figurename~\ref{fig:graphView} del capitolo precedente, potrebbe essere quello riportato nella \figurename~\ref{fig:addText}.
\begin{figure}[!htb]
	\begin{center}
		\includegraphics[width=1 \linewidth]{figure/addText}
	\end{center}
	\caption{aggiunta di commenti da parte dell'utente\label{fig:addText}}
\end{figure}
Nella figura i commenti visualizzati sono a puro titolo di esempio, ma basti pensare alla possibile utilità nel caso in cui si debba lavorare con decine di oggetti magari divisi in sottosezioni differenziate per colore ed in cui ogni raggruppamento ha uno specifico obiettivo che deve essere trasmesso da un utente ad un altro.
\section{navigazione}
La navigazione di una rappresentazione risulta essere una delle principali primitive per la visualizzazione come visto anche nel capitolo riferito proprio alle stesse.\\
Una buona visualizzazione deve poter dare all'utente la possibilità di navigare all'interno del piano di lavoro.\\
Non è però possibile modificare e navigare contemporaneamente in quanto risulterebbe difficoltoso e poco efficiente poter modificare un oggetto quando si sta eseguendo ad esempio una operazione di focus su una particolare sezione di un grafo. Per questo nel momento in cui l'utente ha necessità di eseguire queste operazioni di focus su un particolare elemento del grafo dovrà prima richiedere al sistema di entrare in una modalità che può essere definita "\textit{navigate-view}" in cui ad un primo sguardo non si avranno grandi differenze. Entrando per la prima volta durante una stessa sessione all'interno della \textit{navigate-view} sarà pero visualizzato un messaggio di aiuto per l'utente come mostrato nella figura \figurename~\ref{fig:navMessage} che indicherà le operazioni eseguibili in questa modalità per poter dare anche se in minima parte un consiglio ed un aiuto su come agire.\\
\begin{figure}[!htb]
	\begin{center}
		\includegraphics[width=1 \linewidth]{figure/navMessage}
	\end{center}
	\caption{Messaggio di aiuto al primo accesso dell'utente alla navigate-view\label{fig:navMessage}}
\end{figure}
Spostandosi sul piano di lavoro sarà possibile eseguire le primitive di traslazione e di zooming in/out dell'intera rappresentazione del grafo clusterizzato come mostrato nella \figurename~\ref{fig:zoomGraph} in cui si sono eseguite le operazioni di traslazione e zoom-In rispetto alla rappresentazione della \figurename~\ref{fig:graphView}. Inoltre in questa modalità di visualizzazione passando sopra un oggetto sarà evidenziato a schermo per dare più attenzione e focalizzazione su di esso. In particolare questo focus dipenderà dal tipo di oggetto evidenziato:
\begin{itemize}
	\item \textbf{Nodo}: il raggio $r_n$ viene moltiplicato per una costante e viene visualizzata, in alto a destra, la sua etichetta e l'id del suo cluster di appartenenza;
	\item \textbf{Cluster}: il raggio $r_c$ viene moltiplicato per una costante e viene visualizzata la sua etichetta insieme al livello e alla lista di nodi che possiede al suo interno;
	\item \textbf{Arco}: vengono visualizzati gli id dei nodi che l'arco collega ovvero i suoi attributi \textit{source} e \textit{target}.
\end{itemize}
Gli attributi evidenziati saranno eliminati una volta che l'utente sposterà il cursore verso un oggetto diverso o verso il piano di lavoro di modo da poter lasciare la possibilità al sistema di poter evidenziare un altro oggetto o di proseguire con la sessione di lavoro una volta tolta la \textit{navigate-view}.
\begin{figure}[!htb]
	\begin{center}
		\includegraphics[width=1 \linewidth]{figure/zoomGraph}
	\end{center}
	\caption{zooming di un grafo nella navigate-view\label{fig:zoomGraph}}
\end{figure}
}

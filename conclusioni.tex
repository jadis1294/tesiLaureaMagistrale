\large{
Inizialmente sono stati analizzati e consolidati i problemi appartenenti a ciò che concerne la teoria dei grafi, relativamente ai grafi clusterizzati e alla loro visualizzazione. I problemi principali identificati sono:
\begin{enumerate}
	\item le primitive di integrazione per i grafi non sono utilizzabili per i grafi clusterizzati;
	\item l'assenza delle operazioni di semplificazione all'interno dei visualizzatori; 
	\item la mancanza di software specializzati in grado di trattarne la struttura dati e la conseguente visualizzazione.
\end{enumerate}  
I problemi 1 e 2 sono stati poi risolti mediante uno studio relativo alla creazione di uno "standard" sulle possibili primitive di interazione e sulle operazioni di semplificazione.
Le primitive identificate sono ora utilizzabili per qualunque grafo clusterizzato e visualizzatore. In aggiunta i metodi per la semplificazione, frutto di anni di ricerche, rappresentano un traguardo nell'ambito della ricerca accademica relativa al disegno planare di grafi clusterizzati. La riduzione polinomiale FLAT C\_PLANARITY riportata presenta inoltre un traguardo della ricerca che permette di eseguire indagini sulla complessità della planarità di un grafo clusterizzato legittimamente limitate a grafici flat di grafi planari(indipendenti), trascurando gerarchie più complesse dell'albero di inclusione.	
\newpage
È adesso possibile implementare in qualunque software queste semplificazioni migliorando il processo di ricerca. Lo studio è stato poi accompagnato dalla realizzazione del primo software specialistico in grado di lavorare sui grafi clusterizzati andando dunque a risolvere il problema 3 identificato all'inizio del lavoro svolto. Nonostante i risultati riportati è sempre possibile, essendo ancora un settore poco esplorato, ragionare ed immaginare sviluppi futuri. 
Di seguito è elencata una lista dei possibili sviluppi futuri riguardanti il puro studio teorico:
\begin{itemize}
	\item studio di nuove operazioni di semplificazione; 
	\item definire la complessità del decidere quando un generico grafo clusterizzato $C$ ammette un disegno planare clusterizzato $\tau(C)$;
	\item definizione di nuove primitive di interazione per i grafi clusterizzati
\end{itemize}

Per quanto riguarda invece i visualizzatori ed il software di esempio realizzato possibili sviluppi futuri sono:
\begin{itemize}
	\item creazione di nuovi visualizzatori mediante le primitive di interazione analizzate in questo lavoro;
	\item realizzazione di nuove operazioni di modifica o di creazione a disposizione dell'utente sul software in esempio;
	\item ideare nuove tipologie di visualizzazione da impiegare nel visualizzatori grafici
\end{itemize}
Per concludere la visualizzazione delle informazioni risulta essere un ambito che nel prossimo futuro rivestirà un ruolo di principale importanza in quanto si sta assistendo alla grande crescita relativa alla quantità di dati creati ed elaborati. Essendo poi, nella teoria dei grafi, i problemi riguardanti i grafi clusterizzati ed il disegno planare un ambito di studio aperto da venti anni e ben lontano dalla fine delle ricerche, si riscontra il contributo portato dal lavoro svolto.
}

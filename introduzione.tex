\section{Ringraziamenti}
\label{sec:ringraziamenti}

Lorem ipsum dolor sit amet, consectetur adipisci elit, sed do eiusmod tempor incidunt ut labore et dolore magna aliqua. Ut enim ad minim veniam, quis nostrum exercitationem ullamco laboriosam, nisi ut aliquid ex ea commodi consequatur. Duis aute irure reprehenderit in voluptate velit esse cillum dolore eu fugiat nulla pariatur. Excepteur sint obcaecat cupiditat non proident, sunt in culpa qui officia deserunt mollit anim id est laborum.Lorem ipsum dolor sit amet, consectetur adipisci elit, sed do eiusmod tempor incidunt ut labore et dolore magna aliqua. Ut enim ad minim veniam, quis nostrum exercitationem ullamco laboriosam, nisi ut aliquid ex ea commodi consequatur. Duis aute irure reprehenderit in voluptate velit esse cillum dolore eu fugiat nulla pariatur. Excepteur sint obcaecat cupiditat non proident, sunt in culpa qui officia deserunt mollit anim id est laborum.Lorem ipsum dolor sit amet, consectetur adipisci elit, sed do eiusmod tempor incidunt ut labore et dolore magna aliqua. Ut enim ad minim veniam, quis nostrum exercitationem ullamco laboriosam, nisi ut aliquid ex ea commodi consequatur. Duis aute irure reprehenderit in voluptate velit esse cillum dolore eu fugiat nulla pariatur. Excepteur sint obcaecat cupiditat non proident, sunt in culpa qui officia deserunt mollit anim id est laborum.
   \begin{flushright}
   	\textit{Luca De Silvestris}
   \end{flushright} 
\newpage
\section{Premessa}
Il lavoro che verr\`a descritto in questa tesi rappresenta la relazione finale di un progetto svolto dal candidato nell'ambito della visualizzazione delle informazioni sul tema dei grafi clusterizzati. Il progetto ha riguardato lo sviluppo del primo editor per clustered graphs e con supporto per operazioni di semplificazione e trasformazione e "flattizzazione" del grafo.

Le macro-aree di interesse sono quindi la visualizzazione delle informazioni, la teoria dei grafi, la programmazione web e Javascript. Tutti i concetti enunciati nella descrizione sintetica qui riportata verranno approfonditi, illustrati, documentati e motivati nel corso della lettura. Alla base dello studio vi è la necessità della creazione di un editor che faciliti lo sviluppo di applicazioni grafiche per la ricerca scientifica sui grafi.\\
La tesi si concentra in questi sei capitoli di seguito anticipati:\\ \\
\textbf{1. Stato dell' arte:} breve panoramica sui concetti chiave del lavoro;\\ \\
\textbf{2. Strumenti e metodologie: }un focus sugli strumenti e sulle tecnologie usate per risolvere i problemi e sulle metodologie;\\ \\
\textbf{3. Analisi dei requisiti e problemi:} definizione degli obiettivi e raccolta dei requisiti progettuali e problemi da risolvere individuati durante le fasi di analisi;\\ \\
\textbf{4. Progettazione e implementazione: }la descrizione dell' architettura del progetto finale, con approfondimento delle scelte progettuali;\\ \\
\textbf{5. Esempi di utilizzo e sviluppi futuri:} considerazioni sulle possibili implementazioni future e esempi di applicazione;\\ \\
\textbf{6. Conclusioni: }considerazioni sui risultati ottenuti.
\vspace{0.5cm}

\section{Ringraziamenti}
\label{sec:ringraziamenti}
Una laurea.\\
Una laurea è passione, sacrificio e dedizione. Una laurea è un percorso di crescita che porta alla gloria di un singolo giorno. Eppure nonostante questa consapevolezza, quel singolo giorno di una vita, idealizzato mediante una corona di alloro, ha il sorriso e la gioia della consapevolezza, del traguardo e della vittoria. Del potersi guardare la mattina dopo con uno sguardo diverso. Una laurea è il desiderio di rivalsa con se stessi in quella eterna lotta che si vive cercando una utopistica vittoria con se stessi, che anche se per un solo giorno diventa reale. Detto questo, non è facile andare a ritroso negli anni e rievocare ricordi, emozioni e sensazioni per arrivare ad avere una lista di coloro che mi hanno reso ciò che sono e che spero contribuiranno ad essere ciò che sarò. Desidero ringraziare tutte quelle persone che, con suggerimenti, critiche e osservazioni, hanno fornito un importante aiuto nell'esperienza di questi cinque anni appena trascorsi, con la promessa di migliorarmi sempre.	
Ringrazio mia madre, mio padre e mia sorella a cui ho dedicato questi miei anni di studio, per aver creduto in me fin dall'inizio di questo percorso. Una dedica, in particolare, va a mia sorella Diandra: nella speranza che tu possa sempre gioire dei miei traguardi come io dei tuoi e con la consapevolezza che ti appoggerò sempre, anche se con una visione critica e cinica, in qualunque tua scelta futura.\\
   \begin{flushright}
   	\textit{Luca De Silvestris}
   \end{flushright} 
\newpage
\section{Premessa}
Il lavoro che verr\`a descritto in questa tesi rappresenta la relazione finale di un progetto svolto dal candidato nell'ambito della visualizzazione delle informazioni sul tema dei grafi clusterizzati. Il progetto ha riguardato lo studio delle primitive di interazione e degli strumenti di semplificazione di grafi clusterizzati. Lo studio teorico delle primitive è supportato dallo sviluppo del primo editor per clustered graphs.

Le macro-aree di interesse sono quindi la visualizzazione delle informazioni, la teoria dei grafi le primitive di integrazione e lo sviluppo software web. Tutti i concetti enunciati nella descrizione sintetica qui riportata verranno approfonditi, illustrati, documentati e motivati nel corso della lettura.\\
Il documento si concentra in questi undici capitoli di seguito anticipati:\\
\begin{enumerate}
	\item \textbf{Definizioni preliminari:} breve panoramica sui concetti di base;
	\item \textbf{Stato dell'arte:} Digressione su ciò che attualmente è utilizzabile;
	\item \textbf{Grafi clusterizzati:} definizione dei concetti di studio;
	\item \textbf{Analisi degli obiettivi}
	\item \textbf{Primitive atomiche di interazione:} elenco delle possibili primitive di interazione per grafi clusterizzati
	\item \textbf{Operazioni di semplificazione per grafi clusterizzati }.
	\item \textbf{Editor: interfacce e strutture dati}
	\item \textbf{Editor: interazioni e semplificazioni}
	\item \textbf{Esempio reale di utilizzo }
	\item \textbf{Conclusioni}
\end{enumerate}
% Classe appositamente creata per tesi di Ingegneria Informatica all'università Roma Tre
\documentclass[12pt]{TesiDiaUniroma3}
% --- INIZIO dati relativi al template TesiDiaUniroma3
% dati obbligatori, necessari al frontespizio
\titolo{Editor per grafi clusterizzati con supporto per operazioni di semplificazione e trasformazione}
\autore{Luca De Silvestris}
\matricola{486652}
\relatore{Prof. Maurizio Patrignani}
\correlatore{Prof.  Giuseppe Di Battista} 
\annoAccademico{2018/2019}

% dati opzionali
\dedica{Ai miei genitori} % solo se nel documento si usa il comando \generaDedica
% --- FINE dati relativi al template TesiDiaUniroma3

% --- INIZIO richiamo di pacchetti di utilità. Questi sono un esempio e non sono strettamente necessari al modello per la tesi.
%\usepackage[plainpages=false]{hyperref}	% generazione di collegamenti ipertestuali su indice e riferimenti
%\usepackage[all]{hypcap} % per far si che i link ipertestuali alle immagini puntino all'inizio delle stesse e non alla didascalia sottostante
\usepackage{amsthm}	% per definizioni e teoremi
\usepackage{amsmath}	% per ``cases'' environment
\usepackage{tikz}
\DeclareOldFontCommand{\bf}{\normalfont\bfseries}{\mathbf}
\usetikzlibrary{arrows, calc, chains, positioning}
\tikzset
{
	% nodes
	darkstyle/.style =
	{
		circle,draw,fill = black!20
	},
	% trajectory
	serpent/.style =
	{
		line join  = round,
		line width = 2pt,
		line cap   = round,
		opacity    = .7,
		red
	},
	>=stealth',
	core/.style = 
	{
		rectangle,
		rounded corners,
		draw           = black, thick,
		text width     = 5em, 
		minimum height = 3em,
		align=center,
		on chain
	},
	cache/.style = 
	{
		core,
		draw = gray
	},
	line/.style =
	{
		draw, thick, <-
	},
	element/.style =
	{
		tape,
		top color      = black,
		bottom color   = blue!50!black!60!,
		minimum width  = 8em,
		draw           = blue!40!black!90, very thick,
		text width     = 10em, 
		minimum height = 3.5em, 
		text centered, 
		on chain
	},
	every join/.style =
	{
		->, thick,shorten >=1pt
	},
	%    decoration = 
	%    {
	%        brace
	%    },
	tuborg/.style = 
	{
		decorate
	},
	tubnode/.style = 
	{
		midway, right = 2pt
	},
}

\newcommand{\ie}{\emph{i.e.}}

% --- FINE riachiamo di pacchetti di utilità

\begin{document}
% ----- Pagine di fronespizio, numerate in romano (i,ii,iii,iv...) (obbligatorio)
\frontmatter
\generaFrontespizio
\generaDedica
\generaIndice
\generaIndiceFigure
\generaIndiceTabelle
\introduzione{introduzione}		% inserisce l'introduzione e la prende in questo caso da introduzione.tex


% ----- Pagine di tesi, numerate in arabo (1,2,3,4,...) (obbligatorio)
\mainmatter
% il comando ``capitolo'' ha come parametri:
% 1) il titolo del capitolo
% 2) il nome del file tex (senza estensione) che contiene il capitolo. Tale nome \`e usato anche come label del capitolo
\capitolo{Stato dell'arte}{Capitolo1}
\capitolo{Strumenti e metodologie}{Capitolo2}
\capitolo{Editor e primitive di interazione}{Capitolo4}
\capitolo{Analisi degli obiettivi e scelte progettuali}{Capitolo3}
\capitolo{Interfacce e struttura dati}{Capitolo5}
\capitolo{Funzionalità utente}{Capitolo6}
\capitolo{Riduzione Flat del grafo clusterizzato}{Capitolo7}
\capitolo{Esempi reali di utilizzo}{Capitolo8}
% ----- Parte finale della tesi (obbligatorio)
\backmatter
\conclusioni{conclusioni}

% Bibliografia con BibTeX (obbligatoria)
% Non si deve specificare lo stile della bibliografia
\bibliography{bibliografia} % inserisce la bibliografia e la prende in questo caso da bibliografia.bib

\end{document}
